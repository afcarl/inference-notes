\documentclass[12pt]{article}
\usepackage[margin=1in]{geometry}
\usepackage{epsfig}
\usepackage{amsfonts}
\usepackage[mathcal]{euscript}
\usepackage[tbtags]{amsmath}
\usepackage{amssymb}
\usepackage{mathrsfs}
\usepackage{import}
\usepackage{hyperref}
\usepackage{enumerate}
\usepackage{algorithm}
\usepackage{algorithmicx}
\usepackage[noend]{algpseudocode}
\usepackage{bbm}
\usepackage{subfig}

\renewcommand{\P}{\mathbb{P}}

\newcommand{\example}[3]{%
    \vspace{.15in}

    \noindent \textbf{Example~#1}\nopagebreak

    \noindent #2

    \noindent \textbf{Solution}: #3
}


\DeclareMathOperator*{\argmax}{\mathrm{argmax}}
\renewcommand{\P}{\mathbb{P}}
\newcommand{\E}[1]{\mathbb{E}\left[#1\right]}
\newcommand{\bunderline}[2][4]{\underline{#2\mkern-#1mu}\mkern#1mu}
\newcommand{\phat}{\widehat{p}_X}
\newcommand{\N}{\mathcal{N}}
\newcommand{\indicator}{\mathbbm{1}}

\algrenewcommand\algorithmicrequire{\textbf{Inputs:}}

\title{Gaussian mixture models and the EM algorithm}
\author{Ramesh Sridharan\thanks{Contact: \mbox{rameshvs@csail.mit.edu}}}
\date{}
\begin{document}
    \maketitle
    These notes give a short introduction to Gaussian mixture models (GMMs) and
    the Expectation-Maximization (EM) algorithm, first for the specific case of
    GMMs, and then more generally. These notes assume you're familiar with basic
    probability and basic calculus. If you're interested in the full derivation
    (Section~\ref{sec:EM}), some familiarity with entropy and KL divergence
    is useful but not strictly required.

    The notation here is borrowed from \emph{Introduction to Probability} by
    Bertsekas \& Tsitsiklis: random variables are represented with capital
    letters, values they take are represented with lowercase letters, $p_X$
    represents a probability distribution for random variable $X$, and $p_X(x)$
    represents the probability of value $x$ (according to $p_X$). We'll also
    use the shorthand notation $X_1^{n}$ to represent the sequence $X_1, X_2,
    \ldots, X_n$, and similarly $x_1^n$ to represent $x_1, x_2, \ldots, x_n$.

    These notes follow a development somewhat similar to the one in
    \emph{Pattern Recognition and Machine Learning} by Bishop.


    \section{Review: the Gaussian distribution}
    If random variable $X$ is Gaussian, it has the following PDF:
    \begin{align*}
        p_X(x) &= \frac{1}{\sigma \sqrt{2\pi}} e^{-(x-\mu)^2/2\sigma^2}
    \end{align*}
    The two parameters are $\mu$, the mean, and $\sigma^2$, the variance ($\sigma$
    is called the standard deviation).

    We'll use the terms ``Gaussian'' and ``normal'' interchangeably to refer to
    this distribution.  To save us some writing, we'll write $p_X(x) =
    \mathcal{N}(x; \mu, \sigma^2)$ to mean the same thing (where the
    $\mathcal{N}$ stands for normal).

    \subsection{Parameter estimation for Gaussians: $\mu$}
    \label{sec:gaussian-param}
    Suppose we have i.i.d observations $X_1^n$ from a Gaussian distribution
    with unknown mean $\mu$ and known variance $\sigma^2$. If we want to find
    the maximum likelihood estimate for the parameter $\mu$, we'll find the
    log-likelihood, differentiate, and set it to 0.
    \begin{align*}
        p_{X_1^n}(x_1^n)
            &= \prod_{i=1}^n \N(x_i ; \mu, \sigma^2) 
            = \prod_{i=1}^n \frac{1}{\sigma \sqrt{2\pi}} e^{-(x_1-\mu)^2/2\sigma^2} \\
        \ln p_{X_1^n}(x_1^n)
        &= \sum_{i=1}^n \ln\left(\frac{1}{\sigma \sqrt{2\pi}}\right) - \frac{1}{2\sigma^2}(x_i - \mu)^2 \\
        \frac{d}{d\mu} \ln p_{X_1^n}(x_1^n) 
        &= \sum_{i=1}^n \frac{1}{\sigma^2}(x_i - \mu)
    \end{align*}
    Setting this equal to 0, we see that the maximum likelihood estimate is
    $\widehat{\mu} = \frac{1}{N}\sum_i x_i$: it's the average of our observed
    samples. Notice that this estimate doesn't depend on the variance
    $\sigma^2$! Even though we started off by saying it was known, its value
    didn't matter.

    \section{Gaussian Mixture Models}

    A Gaussian mixture model (GMM) is useful for modeling data that comes from
    one of several groups: the groups might be different from each
    other, but data points within the same group can be well-modeled by a
    Gaussian distribution.

    \subsection{Examples}
    For example, suppose the price of a randomly chosen paperback book is
    normally distributed with mean $\$10.00$ and standard deviation $\$1.00$.
    Similarly, the price of a randomly chosen hardback is normally distributed
    with mean $\$17$ and variance $\$1.50$. Is the price of a randomly chosen
    book normally distributed?

    The answer is no. Intuitively, we can see this by looking at the
    fundamental property of the normal distribution: it's highest near the
    center, and quickly drops off as you get farther away.  But, the
    distribution of a randomly chosen book is bimodal: the center of the
    distribution is near $\$13$, but the probability of finding a book near
    that price is lower than the probability of finding a book for a few
    dollars more or a few dollars less. This is illustrated in
    Figure~\ref{fig:gmm-books}.

    Another example: the height of a randomly chosen man is normally
    distributed with a mean around $5'9.5"$ and standard deviation around
    $2.5"$. Similarly, the height of a randomly chosen woman is normally
    distributed with a mean around $5'4.5"$ and standard deviation around
    $2.5"$~\footnote{In the metric system, the means are about 177 cm and 164 cm,
    and the standard deviations are about 6 cm.} Is the height of a randomly
    chosen person normally distributed?

    The answer is again no. This one is a little more deceptive: because
    there's so much overlap between the height distributions for men and for
    women, the overall distribution is in fact highest at the center. But it's
    still not normally distributed: it's too wide and flat in the center (we'll
    formalize this idea in just a moment). This is illustrated in
    Figure~\ref{fig:gmm-heights}.
        \begin{figure}[t]
            \centering
            \hspace*{\fill}
            \subfloat[Probability density for paperback books (red), hardback books (blue),
                and all books (black, solid)]{%
                \includegraphics{gmm-books}
                \label{fig:gmm-books}}
            \hfill
            \subfloat[Probability density for heights of women (red), heights of men (blue),
                and all heights (black, solid)]{%
                    \includegraphics{gmm-heights}
                    \label{fig:gmm-heights}}
            \caption{Two Gaussian mixture models: the component densities
                (which are Gaussian) are shown in dotted red and blue lines,
                while the overall density (which is not) is shown as a solid
                black line.}
            \hspace*{\fill}
        \end{figure}
    These are both examples of \emph{mixtures of Gaussians}: distributions
    where we have several groups and the data within each group is normally
    distributed.  Let's look at this a little more formally with heights.

    \subsection{The model}
    Formally, suppose we have people numbered $i=1, \ldots, n$. We observe
    random variable $Y_i \in \mathbb{R}$ for each person's height, and assume
    there's an unobserved label $C_i \in \{M, F\}$ for each person
    representing that person's gender~\footnote{% If you're familiar with the
        Naive Bayes model, this is somewhat similar. However, here our features
        are always Gaussian, and in the general case of more than 1 dimension,
    we won't assume independence of the features.}. Here, the letter $c$ stands
    for ``class''.  In general, we can have any number of possible labels or
    classes, but we'll limit ourselves to two for this example. We'll also
    assume that the two groups have the same known variance $\sigma^2$, but
    different unknown means $\mu_M$ and $\mu_F$.  The distribution for the
    class labels is Bernoulli:
    \begin{align*}
        p_{C_i}(c_i) &= q^{\indicator(c_i = M)}(1-q)^{\indicator(c_i = F)}
    \end{align*}
    We'll also assume $q$ is known. To simplify notation later, we'll let
    $\pi_M = q$ and $\pi_F = 1-q$, so we can write
    \begin{align}
        \label{eq:pc} p_{C_i}(c_i) &= \prod_{c \in \{M, F\}} \pi_c^{\indicator(c_i = c)}
    \end{align}

    The conditional distributions within each class are Gaussian:
    \begin{align}
        \label{eq:pygivenc} p_{Y_i|C_i}(y_i | c_i)
            &= \prod_{c} \N(y_i ; \mu_c , \sigma^2)^{\indicator(c_i = c)}
    \end{align}
    \subsection{Parameter estimation: a first attempt}
    \label{sec:parameter-estimation-attempt}
    Suppose we observe i.i.d.\ heights $Y_1 = y_1, \ldots, Y_n = y_n$, and we want to
    find maximum likelihood estimates for the parameters $\mu_M$ and $\mu_F$.
    This is an \emph{unsupervised learning} problem: we don't get to observe
    the male/female labels for our data, but we want to learn parameters based
    on those labels~\footnote{Note that in a truly unsupervised setting, we
        wouldn't be able to tell which one of the two was male and which was
        female: we'd find two distinct clusters and have to label them based on
        their values after the fact.}

    \noindent \textbf{Exercise}: Given the model setup in~\eqref{eq:pc}
    and~\eqref{eq:pygivenc}, compute the joint density of all the data points
    $p_{Y_1, \ldots, Y_N}(y_1, \ldots, y_n)$ in terms of $\mu_M$, $\mu_F$,
    $\sigma$, and $q$. Take the log to find the log-likelihood, and then
    differentiate with respect to $\mu_M$. Why is this hard to optimize?

    \noindent \textbf{Solution}: We'll start with the density for a single data
    point $Y_i = y_i$:
    \begin{align*}
        p_{Y_i}(y_i) 
        &= \sum_{c_i} p_{C_i}(c_i) p_{Y_i|C_i}(y_i | c_i) \\
        &= \sum_{c_i} \left(\pi_c \N(y_i; \mu_C, \sigma^2)\right)^{\indicator(c_i = c)} \\
        &= q\N(y_i; \mu_M, \sigma^2) + (1-q)\N(y_i; \mu_F, \sigma^2)
    \end{align*}
    Now, the joint density of all the observations is:
    \begin{align*}
        p_{Y_1^n}(y_1^n) 
        &= \prod_{i=1}^n \left(q\N(y_i; \mu_M, \sigma^2) + (1-q)\N(y_i; \mu_F, \sigma^2) \right),
    \end{align*}
    and the log-likelihood of the parameters is then
    \begin{align}
        \label{eq:log-likelihood-marginalized} \ln p_{Y_1^n}(y_1^n) 
        &= \sum_{i=1}^n \ln\left(\pi_M\N(y_i; \mu_M, \sigma^2) + \pi_F\N(y_i; \mu_F, \sigma^2) \right),
    \end{align}
    We've already run into a small snag: the sum prevents us from applying the
    log to the normal densities inside. So, we should already be a little worried that
    our optimization won't go as smoothly as it did for the simple mean estimation we
    did back in Section~\ref{sec:gaussian-param}. By symmetry, we only need to look at one
    of the means; the other will follow almost the same process. Before we dive into
    differentiating, we note that
    \begin{align*}
        \frac{d}{d\mu} \N(x; \mu, \sigma^2) 
        &= \frac{d}{d\mu} \left[\frac{1}{\sigma\sqrt{2\pi}} e^{-\frac{(x-\mu)^2}{2\sigma^2}} \right] \\
        &=  \frac{1}{\sigma\sqrt{2\pi}} e^{-\frac{(x-\mu)^2}{2\sigma^2}} \cdot \frac{2(x-\mu)}{2\sigma^2} \\
        &=  \mathcal{N}(x ; \mu, \sigma^2) \cdot \frac{(x-\mu)}{\sigma^2}
    \end{align*}
    Differentiating~\eqref{eq:log-likelihood-marginalized} with respect to $\mu_M$, we obtain
    \begin{align}
        \label{eq:derivative}
            \sum_{i=1}^n \frac{1}{\pi_M\N(y_i; \mu_M, \sigma^2) + \pi_F\N(y_i; \mu_F, \sigma^2)} \pi_M \N(y_i; \mu_M, \sigma^2) \frac{y_i-\mu_M}{\sigma^2} &= 0
    \end{align}
    At this point, we're stuck. We have a mix of ratios of exponentials and
    linear terms, and there's no way we can solve this in closed form to get a
    clean maximum likelihood expression!

    \subsection{Using hidden variables and the EM Algorithm}
    Taking a step back, what would make this computation easier? If we knew the
    hidden labels $C_i$ exactly, then it would be easy to do ML estimates for
    the parameters: we'd take all the points for which $C_i = M$ and use those
    to estimate $\mu_M$ like we did in Section~\ref{sec:gaussian-param}, and
    then repeat for the points where $C_i = F$ to estimate $\mu_F$.  Motivated
    by this, let's try to compute the distribution for $C_i$ given the
    observations. We'll start with Bayes' rule:
    \begin{align}
        \notag p_{C_i | Y_i}(c_i | y_i) &= \frac{p_{Y_i | C_i}(y_i | c_i) p_{C_i}(c_i)}{p_{Y_i}(y_i)} \\
        \label{eq:qc-general} &= \frac{ \prod_{c \in \{M, F\}} \left(\pi_c \N(y_i ; \mu_c, \sigma^2)\right)^{\indicator(c = c_i)}}{\pi_M\N(y_i; \mu_M, \sigma^2) + \pi_F\N(y_i; \mu_F, \sigma^2)} = q_{C_i}(c_i) \\
        \intertext{Let's look at the posterior probability that $C_i = M$:}
        \label{eq:qc} p_{C_i | Y_i}(M | y_i) &= \frac{\pi_M \N(y_i ; \mu_M, \sigma^2)}{\pi_M\N(y_i; \mu_M, \sigma^2) + \pi_F\N(y_i; \mu_F, \sigma^2)} = q_{C_i}(M)
    \end{align}
    This should look very familiar: it's one of the terms
    in~\eqref{eq:derivative}!  And just like in that equation, we have to know
    all the parameters in order to compute this too. We can
    rewrite~\eqref{eq:derivative} in terms of $q_{C_i}$, and cheat a little by
    pretending it doesn't depend on $\mu_M$:
    \begin{equation}
        \label{eq:pre-m-step} \sum_{i=1}^n q_{C_i}(M) \frac{y_i-\mu_M}{\sigma^2} = 0 \\
    \end{equation}
    \begin{align}
        \label{eq:weighted-average} \mu_M &= \frac{\displaystyle \sum_{i=1}^n q_{C_i}(M) y_i}{\displaystyle \sum_{i=1}^n q_{C_i}(M)}
    \end{align}
    This looks much better: $\mu_M$ is a weighted average of the heights, where each
    height is weighted by how likely that person is to be male. By symmetry, for
    $\mu_F$, we'd compute the weighted average with weights $q_{C_i}(F)$.

    So now we have a circular setup: we could easily compute the posteriors
    over $C_1^n$ if we knew the parameters, and we could easily estimate the
    parameters if we knew the posterior over $C_1^n$. This naturally suggests
    the following strategy: we'll fix one and solve for the other.  This
    approach is generally known as the \emph{EM algorithm}. Informally, here's
    how it works:
    \begin{itemize}
        \item First, we fix the parameters (in this case, the means $\mu_M$ and $\mu_F$ of
            the Gaussians) and solve for the posterior distribution for the hidden
            variables (in this case, $q_{C_i}$, the class labels). This is done
            using~\eqref{eq:qc}.
        \item Then, we fix the posterior distribution for the hidden variables
            (again, that's $q_{C_i}$, the class labels), and optimize the
            parameters (the means $\mu_M$ and $\mu_F$) using the expected
            values of the hidden variables (in this case, the probabilities
            from $q_{C_i}$). This is done using~\eqref{eq:derivative}.
        \item Repeat the two steps above until the values aren't changing much
            (i.e., until convergence).
    \end{itemize}
    Note that in order to get the process started, we have to initialize the
    parameters somehow. In this setting, the initialization matters a lot! For
    example, suppose we set $\mu_M = 3'$ and $\mu_F = 5'$. Then the computed
    posteriors $q_{C_i}$ would all favor $F$ over $M$ (since most people are
    closer to $5'$ than $3'$), and we would end up computing $\mu_F$ as roughly
    the average of all our heights, and $\mu_M$ as the average of a few short
    people.

    \pagebreak
    \section{The EM Algorithm: a more formal look}
    \label{sec:EM}

    \paragraph{\textbf{Note:}} This section assumes you have a basic
    familiarity with measures like entropy and KL divergence, and how they
    relate to expectations of random variables. You can still understand the
    algorithm itself without knowing these concepts, but the derivations depend
    on understanding them.

    By this point you might be wondering what the big deal is: the algorithm
    described above may sound like a hack where we just arbitrarily fix some
    stuff and then compute other stuff. But, as we'll show in a few short
    steps, the EM algorithm is actually maximizing a lower bound on the log
    likelihood (in other words, each step is guaranteed to improve our answer
    until convergence). A bit more on that later, but for now let's look at how
    we can derive the algorithm a little more formally.

    Suppose we have observed a random variable $Y$.  Now suppose we also have
    some hidden variable $C$ that $Y$ depends on. Let's say that the
    distributions of $C$ and $Y$ have some parameters $\theta$ that we don't
    know, but are interested in finding.

    In our last example, we observed heights $Y = \{Y_1, \ldots, Y_n\}$ with
    hidden variables (gender labels) $C = \{C_1, \ldots, C_n\}$ (with i.i.d.\
    structure over $Y$ and $C$), and our parameters $\theta$ were $\mu_M$ and
    $\mu_F$, the mean heights for each group.

    Before we can actually derive the algorithm, we'll need a key fact: Jensen's
    inequality. The specific case of Jensen's inequality that we need
    says that:
    \begin{align}
        \label{eq:jensen-log}\log(\mathbb{E}[X]) \geq \mathbb{E}[\log(X)]
    \end{align}
    For a geometric intuition of why this is true, see Figure~\ref{fig:jensen}.
    For a proof and more detail, see
    Wikipedia~\footnote{\url{http://en.wikipedia.org/wiki/Jensen_inequality}}~\footnote{This
        figure is based on the one from the Wikipedia article, but for a
    concave function instead of a convex one.}.
    \begin{figure}[t]
        \centering
            \hspace*{\fill}
            \includegraphics{px_only.pdf}
            \hfill
            \includegraphics{px_log.pdf}
            \hfill
            \includegraphics{px_py_full.pdf}
            \hspace*{\fill}
        \caption{An illustration of a special case of Jensen's inequality:
            for any random variable $X$, $\mathbb{E}[\log X] \geq \log \mathbb{E}[X]$. Let
            $X$ be a random variable with PDF as shown in red. Let
            $Z=\log X$. The center and right figures show how to construct
            the PDF for $Z$ (shown in blue): because of the log, it's skewed towards
            smaller values compared to the PDF for $X$. $\log\mathbb{E}[X]$ is the point given by the
            center dotted black line, or $\E{X}$. But $\E{\log X}$, or 
            $\E{Z}$, will always be smaller (or at least will never be larger) because the log ``squashes''
            the bigger end of the distribution (where $Z$ is larger) and ``stretches'' the smaller
            end (where $Z$ is smaller).}
        \label{fig:jensen}
    \end{figure}

    Now we're ready to begin: Section~\ref{sec:em-short} goes through the
    derivation quickly, and Section~\ref{sec:em-long} goes into more detail
    about each step.
    \subsection{The short verion}
    \label{sec:em-short}
    We want to maximize the log-likelihood:
    \begin{align*}
        \log p_Y(y; \theta) \\
        \scriptsize{\text{(Marginalizing over $C$ and introducing $q_C(c)/q_C(c)$)}}\quad
            &= \log\left(\sum_c q_C(c) \frac{p_{Y,C}(y,c; \theta)}{q_C(c)}\right) \\
            \scriptsize{\text{(Rewriting as an expectation)}}\quad
            &= \log\left(\mathbb{E}_{q_C}\left[\frac{p_{Y,C}(y,C; \theta)}{q_C(C)}\right]\right)\\
            \scriptsize{\text{(Using Jensen's inequality)}}\quad
            &\geq \mathbb{E}_{q_C}\left[\log \frac{p_{Y,C}(y,C; \theta)}{q_C(C)} \right]
    \end{align*}
    Let's rearrange the last version:
    \begin{align*}
        \mathbb{E}_{q_C}\left[\log \frac{p_{Y,C}(y,C; \theta)}{q_C(C)} \right]
        &= \mathbb{E}_{q_C}\left[\log p_{Y,C}(y,C; \theta)\right] - \mathbb{E}_{q_C} [\log q_C(C)]
    \end{align*}
    Maximizing with respect to $\theta$ will give us:
    \begin{equation*}
        \boxed{\widehat{\theta} \gets \argmax_\theta \mathbb{E}_{q_C}\left[\log p_{Y,C}(y,C; \theta)\right]}
    \end{equation*}
    That's the M-step. Now we'll rearrange a different way:
    \begin{align*}
        \mathbb{E}_{q_C}\left[\log \frac{p_{Y,C}(y,C; \theta)}{q_C(C)} \right] 
        &= \mathbb{E}_{q_C}\left[\log \frac{p_{Y}(y; \theta)p_{C|Y}(C|y; \theta)}{q_C(C)} \right] \\
        &= \log p_{Y}(y; \theta) - \mathbb{E}_{q_C}\left[\log \frac{q_C(C)}{p_{C|Y}(C|y; \theta)}\right] \\
        &= \log p_{Y}(y; \theta) - D(q_C(\cdot)||p_{C|Y}(\cdot|y; \theta))
    \end{align*}
    Maximizing with respect to $q_C$ will give us:
    \begin{equation*}
        \boxed{\widehat{q}_C(\cdot) \gets p_{C|Y}(\cdot|y; \theta)}
    \end{equation*}
    That's the E-step.

    \subsection{The long version}
    \label{sec:em-long}
    We'll try to do maximum likelihood. Just like we did earlier, we'll try to
    compute the log-likelihood by marginalizing over $C$:
    \begin{align}
        \notag \log p_Y(y; \theta) 
        &= \log\left(\sum_c p_{Y,C}(y,c)\right)
    \end{align}
    Just like in Section~\ref{sec:parameter-estimation-attempt}, we're stuck
    here: we can't do much with a log of a sum.  Wouldn't it be nice if we
    could swap the order of them? Well, an expectation is a special kind of
    sum, and Jensen's inequality lets us swap them if we have an expectation.
    So, we'll introduce a new distribution $q_C$ for the hidden variable $C$:
    \begin{align}
        \log p_Y(y; \theta) \\
        \notag \scriptsize{\text{(Marginalizing over $C$ and introducing $q_C(c)/q_C(c)$)}}\quad
            &= \log\left(\sum_c q_C(c) \frac{p_{Y,C}(y,c; \theta)}{q_C(c)}\right) \\
        \notag \scriptsize{\text{(Rewriting as an expectation)}}\quad
            &= \log\left(\mathbb{E}_{q_C}\left[\frac{p_{Y,C}(y,C; \theta)}{q_C(C)}\right]\right) \\
        \label{eq:em-core-theta} \scriptsize{\text{(Using Jensen's inequality)}}\quad &\geq \mathbb{E}_{q_C}\left[\log \frac{p_{Y,C}(y,C; \theta)}{q_C(C)} \right] \\
        \label{eq:em-core-q} \scriptsize{\text{Using definition of conditional probability}}\quad &= \mathbb{E}_{q_C}\left[\log \frac{p_{Y}(y; \theta)p_{C|Y}(C|y; \theta)}{q_C(C)} \right]
    \end{align}
    Now we have a lower bound on $\log p_Y(y; \theta)$ that we can optimize
    pretty easily. Since we've introduced $q_C$, we now want to maximize
    this quantity with respect to both $\theta$ and $q_C$.

    We'll use~\eqref{eq:em-core-theta} and~\eqref{eq:em-core-q}, respectively, to
    do the optimizations separately. First, using~\eqref{eq:em-core-theta} to find the
    best parameters:
    \begin{align*}
        \mathbb{E}_{q_C}\left[\log \frac{p_{Y,C}(y,C; \theta)}{q_C(C)} \right]
        &= \mathbb{E}_{q_C}\left[\log p_{Y,C}(y,C; \theta)\right] - \mathbb{E}_{q_C} [\log q_C(C)]
    \end{align*}
    In general, $q_C$ doesn't depend on $\theta$, so we'll only care about the first term:
    \begin{equation}
        \label{eq:m-step} \boxed{\widehat{\theta} \gets \argmax_\theta \mathbb{E}_{q_C}\left[\log p_{Y,C}(y,C; \theta)\right]}
    \end{equation}
    This is called the \emph{M-step}: the M stands for maximization, since we're
    maximizing with respect to the parameters.
    Now, let's find the best $q_C$ using~\eqref{eq:em-core-q}.
    \begin{align}
        \notag \mathbb{E}_{q_C}\left[\log \frac{p_{Y}(y; \theta)p_{C|Y}(C|y; \theta)}{q_C(C)} \right]
        &= \mathbb{E}_{q_C}[\log p_{Y}(y; \theta)] + \mathbb{E}_{q_C}\left[\log \frac{p_{C|Y}(C|y; \theta)}{q_C(C)}\right] \\
        \intertext{The first term doesn't depend on $c$, and the second term almost looks like a KL divergence:}
        \notag &= \log p_{Y}(y; \theta) - \mathbb{E}_{q_C}\left[\log \frac{q_C(C)}{p_{C|Y}(C|y; \theta)}\right] \\
        \label{eq:e-step-divergence} &= \log p_{Y}(y; \theta) - D(q_C(\cdot)||p_{C|Y}(\cdot|y; \theta))~\footnotemark
    \end{align}
        \footnotetext{Remember that this is a lower bound
            on $\log p_Y(y; \theta)$: that is, $\log p_Y(y; \theta) \geq \log p_{Y}(y; \theta) - D(q_C(\cdot)||p_{C|Y}(\cdot|y; \theta))$.
            From this, we can see that the ``gap'' in the lower bound comes entirely from the KL divergence term.}
    So, when maximizing this quantity, we want to make the KL divergence as
    small as possible. KL divergences are always greater than or equal to 0,
    and they're exactly 0 when the two distributions are equal. So, the optimal
    $q_C$ is $p_{C|Y}(c|y; \theta)$:
    \begin{equation}
        \label{eq:e-step} \boxed{\widehat{q}_C(c) \gets p_{C|Y}(c|y; \theta)}
    \end{equation}
    This is called the \emph{E-step}: the E stands for expectation, since we're
    computing $q_C$ so that we can use it for expectations.
    So, by alternating between~\eqref{eq:m-step} and~\eqref{eq:e-step}, we can
    maximize a lower bound on the log-likelihood. We've also seen from~\eqref{eq:e-step}
    that the lower bound is tight (that is, it's equal to the log-likelihood)
    when we use~\eqref{eq:e-step}.

    \subsection{The algorithm}

    \noindent \hrulefill
        \begin{algorithmic}[1]
            \Require{Observation $y$, joint distribution $p_{Y,C}(y,c; \theta)$, conditional distribution
                $p_{C|Y}(c|y; \theta)$, initial values $\theta^{(0)}$}
            \Statex
            \Function{EM}{$p_{Y,C}(y,c; \theta), p_{C|Y}(c|y; \theta), \theta^{(0)}$}
                \For{iteration $t \in 1,2,\ldots$}
                    \State{$q_C^{(t)} \gets p_{C|Y}(c|y; \theta^{(t-1)})$ \quad (\textbf{E-step})}
                    \State{$\theta^{(t)} \gets \argmax_\theta \mathbb{E}_{q^{(t)}_C} \left[p_{Y,C}(y, C; \theta)\right]$\quad (\textbf{M-step})}
                    \If{$\theta^{(t)} \approx \theta^{(t-1)}$}
                        \State{return $\theta^{(t)}$}
                    \EndIf
                \EndFor
            \EndFunction
        \end{algorithmic}
    \hrulefill

    \subsection{Example: Applying the general algorithm to GMMs}
    Now, let's revisit our GMM for heights and see how we can apply the two
    steps. We have the observed variable $Y = \{Y_1, \ldots, Y_n\}$, and the
    hidden variable $C = \{C_1, \ldots, C_n\}$. For the E-step, we have to
    compute the posterior distribution $p_{C|Y}(c|y)$, which we already did
    in~\eqref{eq:qc-general} and~\eqref{eq:qc}. For the M-step, we have to
    compute the expected joint probability.
    \begin{align*}
        \mathbb{E}_{q_C}[\ln p_{Y,C}(y,C)]
        &= \mathbb{E}_{q_C}[\ln p_{Y|C}(y|C)p_C(C)] \\
        &= \mathbb{E}_{q_C}\left[\ln \prod_{i=1}^n \prod_{c \in \{M, F\}} \left(\pi_c\N(y_i; \mu_c, \sigma^2)\right)^{\indicator(C_i = c)}\right] \\
        &= \mathbb{E}_{q_C}\left[ \sum_{i=1}^n \sum_{c \in \{M, F\}} \indicator(C_i = c) \left(\ln \pi_c + \ln \N(y_i; \mu_c, \sigma^2)\right)\right] \\
        &=  \sum_{i=1}^n \sum_{c \in \{M, F\}} \mathbb{E}_{q_C}\left[\indicator(C_i = c)\right] \left(\ln \pi_c + \ln \frac{1}{\sigma\sqrt{2\pi}} - \frac{(y_i-\mu_c)^2}{2\sigma^2} \right)
    \end{align*}
    $\mathbb{E}_{q_C}\left[\indicator(C_i = c)\right]$ is the
    probability that $C_i$ is $c$, according to $q$. Now, we can differentiate
    with respect to $\mu_M$:
    \begin{align*}
        \frac{d}{d\mu_M} \mathbb{E}_{q_C}[\ln p_{Y|C}(y|C)p_C(C)]
        &= \sum_{i=1}^n q_{C_i}(M) \left(\frac{y_i-\mu_M}{\sigma^2}\right) = 0
    \end{align*}
    This is exactly the same as what we found earlier in~\eqref{eq:pre-m-step},
    so we know the solution will again be the weighted average
    from~\eqref{eq:weighted-average}:
    \begin{align*}
        \mu_M &= \frac{\displaystyle \sum_{i=1}^n q_{C_i}(M) y_i}{\displaystyle \sum_{i=1}^n q_{C_i}(M)}
    \end{align*}
    Repeating the process for $\mu_F$, we get
    \begin{align*}
        \mu_F &= \frac{\displaystyle \sum_{i=1}^n q_{C_i}(F) y_i}{\displaystyle \sum_{i=1}^n q_{C_i}(F)}
    \end{align*}
\end{document}
